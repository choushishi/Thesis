%!TEX root = main.tex

\chapter{Conclusion}

This thesis successfully fulfilled the required tasks in the research objectives and question.

To assist the design of a long-span bridge of Iv-infra, a simplified model for assessing lateral bridge resonance behavior is developed in this thesis. This model is validated to be conservative and reasonable by benchmark. 

However, due to the lack of data available in creating the model, the model is not validated to be applied universally on real-life project. Currently it has following disadvantages:

\begin{enumerate}
	\item Only one concentrated force is modeled to represent the lateral dynamic effect induced by railway vehicle. It means the load in the model can not represent the distribution of vehicle axle forces. 
	\item Amplitude $Q$ is calculated based on a specifically chosen numerical simulation case. However, the aim of the model is to generically simulate vehicle-bridge response behavior, and such specifically choosing may be against this principle of generic.  
	\item The model is not fully validated because of the small quantity of available simulation results for validation. These simulation scenarios can not represent generic real-life scenarios.
	\item The model is not calibrated for modern Dutch railway because model parameter $Q$ is based on data generated by old railway vehicles.\footnote{Simulations\citep{d181dt329} were conducted during 1990s using parameters extracted from real trains at the time. Compared to trains of 1990s, modern railway vehicles possess more sophisticated suspension systems designed to suppress the lateral motion of the vehicle thus they are expected to induce lower lateral forces to tracks.} 
	\item The longest bridge in numerical simulation is 120m long. Thus the model is not validated for bridges longer than 120m.
\end{enumerate}

Despite the disadvantages of the current model, it provides a direction of analyzing lateral dynamics of railway bridges which is different from nowadays available analyzing techniques. It offers a simple approach to avoid heavy numerical simulations during the analysis and therefore, saves the effort and cost for designers. The model shall be regarded as a prototype that can be improved and expanded by future researches. See Chapter.\ref{sec:recommendations} for details of recommendations for future researches.