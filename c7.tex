%!TEX root = main.tex

\chapter{Recommendations for future researches}\label{sec:recommendations}

The recommendations aims to guide future researches to be conducted according to the disadvantages mentioned in previous chapter.

General recommendations:

\begin{enumerate}
	\item Improve the model by make more sophisticated assumption to better reflect real-life scenario. For example: use more than one concentrated force to represent the lateral dynamic load. 
	\item It is recommended to use a larger database when determining the expression for parameters.
	\item Conduct more numerical simulations to provide database for the sake of creation and validation of the model. See Section.\ref{sec:recommendationsonsimulations} for detailed explanation on how to conduct these simulations.
\end{enumerate}

\section{Recommendations on numerical simulations}\label{sec:recommendationsonsimulations}

More accurate statistical result can be yielded with more simulation data. Since now only 6 sets of data are used, the simplified model is not globally reliable. However, it is recommended that future research uses a larger simulation data base to further improve the accuracy of the model.

It is possible to modify the amplitude of the model to a less conservative value according to newly conducted simulations. It is expected that newly conducted simulations yield smaller lateral force on tracks because of the advanced suspension systems implemented in modern vehicle designs and better track quality.

Therefore, numerical simulations are recommended to be conducted according to following suggestions:

\begin{enumerate}
    \item Use more realistic and up-to-date data on modern Dutch train vehicles and railway. The result will help the model to be applicable for Dutch bridges.
    \item Investigate over a broader range of bridge span(greater than 150m).
\end{enumerate}

