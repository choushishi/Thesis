%!TEX root = first try.tex

\chapter{Subject Proposal}

The lateral dynamic behaviour of steel railway bridges are seldom discussed in Eurocodes.  There is one criterion in Eurocode requiring railway bridges should have a lateral natural frequency higher than 1.2Hz. However, this criterion is becoming more and more unsuitable because longer span provides lower natural frequencies. For bridges having span more than 100m, it is almost guaranteed that the first lateral natrual frequency of the bridge fall below 1.2Hz, unable to meet the requirement of Eurocode.

In order to provide a better tool for designers when they encounter lateral dynamic problems on steel railway bridges, following objectives are made:

\begin{enumerate}

\item Study of the background of the criteria of railway bridges, rails and train vehicles in order to give a better understanding of the background of the criteria which is unclear to the designers.  Create an inventory of criteria on lateral dynamic by selecting criteria among systems mentioned above. 

\item Create simplified model for easier analysing interaction of train vehicle and bridge structure based on criteria inventory made in previous step. The model should be representative for original system and in cooperate with criteria inventory.

Several hypothesis have been made and will be illustrated during kick-off presentation.

\item Verify the model developed in the previous step by designing a long-span railway bridge. 

\end{enumerate}