%!TEX root = main.tex


\chapter{Introduction}

\section{Context of the thesis}
This master thesis is initiated by Iv-Infra to assist the designing of a long-span railway bridge. 

During the validation process of bridge lateral dynamics, the bridge could not meet the Eurocode standard, that all railway bridges should possess first lateral natural frequency higher than $1.2Hz$. However, for long-span(longer than $100m$) bridges with adequate structural safety, they normally possess first lateral natural frequencies that are below $1.0Hz$. It is costly and excessive wasting to increase their first lateral natural frequency to $1.2Hz$. Moreover, for the current stage of the project, it is not possible to modify the design of the bridge. Thus increasing the natural frequency of the bridge to meet the criterion is not a valid solution.

Moreover, there is no further instruction to guide the design when bridge can not meet this particular criterion. With no option left within the codes, Iv-Infra seeks alternative assessment for the safety of lateral dynamics of the railway bridge by initiating this thesis. 

\section{Lateral dynamics of railway bridge}
Lateral dynamics of railway bridge is an engineering topic that relates to both bridge structure and operating railway vehicles. Till now there is no record of bridge/vehicle failure due to lateral railway vehicle dynamic loading. 

There are few researches done on this topic. European Rail Research Institute(ERRI), former International Union of Railways(UIC), had systematically investigated this topic in 1994. Several criteria proposed by the investigation were adopted in Eurocode. 

\section{Objectives and research question}\label{sec:introduction}

The main objective of this thesis is to help Iv-Infra to assess the lateral dynamics of railway bridge. In order to do so, the principle of assessing lateral dynamics must be comprehended first. 

Eurocode has offered assessing methods in its criteria related to lateral dynamics of railway bridges. However, there's no additional explanation for these criteria. In other words, the principle of the assessing methods is uncertain. Thus, the first objective of this thesis is to analyze the criteria in Eurocode, in order to interpret the principle of Eurocode assessing methods. By using the principle interpreted in previous step, the development of an alternative assessing method can be inspired and guided. 

As a conclusion, the objectives of this thesis can be summarized as follows:

\begin{enumerate}[-]

\item Analyze Eurocode criteria and interpret the principle of lateral dynamics assessing process
\item Find an alternative method to assess the lateral dynamics of railway bridges

\end{enumerate}

\vspace*{0.5cm}

The research question can be summarized as:

\vspace*{0.5cm}

\emph{Interpret the principle of lateral dynamics validation process and develop an alternative method to assess the lateral dynamics of railway bridges}


\section{Main steps}

In order to carry out the objectives and research questions, the thesis project is planned to be conducted in following steps:

\begin{enumerate}

\item Literature research on the theory of lateral railway bridge dynamics

\item Filter out Eurocode criteria related to lateral dynamics of railway bridges and trace the origin of these criteria

\item Analyze these criteria and conclude the principle of Eurocode assessing method

\item Develop an alternative method for assessing the lateral dynamics of railway bridges

\item Use the alternative method to validate the bridge design of Iv-Infra

\end{enumerate}

\section{Outline of the report}
The report is consists of 3 main parts: \textbf{Introduction}, \textbf{Body} and \textbf{Conclusion}.

\textbf{Introduction} contains Context of the thesis, Brief introduction to the topic, Research objectives and question and Outline of the report.

\textbf{Body} is consisted of three parts. The first part contains literature research on the theory of lateral railway bridge dynamics. The second part mainly aims to describe the analysis process and the conclusion of the analysis while the third part aims to describe the development of the alternative assessing method.

\textbf{Conclusion} contains conclusions for the whole thesis and recommendations for whom may concern in the future. 

Thus the outline can be concluded as follows:

\begin{enumerate}
    \item \textbf{Introduction}
    \begin{enumerate}
        \item[a.] Context of the thesis
        \item[b.] Brief introduction to the topic
        \item[c.] Research objectives and question
        \item[d.] Outline of the report
    \end{enumerate}
    \item \textbf{Body}
    \begin{enumerate}
        \item[a.] Literature research on the theory of lateral railway bridge dynamics
        \item[b.] Analysis of Eurocode criteria
        \item[c.] Development of alternative assessing method 
    \end{enumerate}
    \item \textbf{Conclusion}
    \begin{enumerate}
        \item[a.] Conclusions
        \item[b.] Recommendations
    \end{enumerate} 
\end{enumerate}