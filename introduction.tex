%!TEX root = first try.tex

\chapter{INTRODUCTION}

\section{Summary of topic}

The lateral dynamic behaviour of steel railway bridges are minimally discussed in Eurocodes and designers lack knowledge of background of criteria proposed in the code. For example, there is one criterion in Eurocode requiring railway bridges should have a lateral natural frequency higher than 1.2Hz. However, this criterion is becoming more and more unsuitable because longer span provides lower natural frequencies. For bridges having span more than 100m, it is almost guaranteed that the first lateral natural frequency of the bridge fall below 1.2Hz, unable to meet the requirement of Eurocode. 

Criteria on lateral dynamics of railway bridges are complicated if taking vehicle systems and interaction into account. Designers need a better knowledge on railway dynamics and a tool in calculating the lateral dynamic behaviour of the whole system. This tool needs to be simple to meet the engineering needs.  

\section{Motivation of the thesis}


\section{Objectives and research question}\label{sec:introduction}

% The main goal is to think of a method to verify whether a bridge is expected to encounter transverse dynamic problems. 

% In the literature study the criteria of different systems involved in dynamic response of steel railway bridges will be investigated. Governing criteria will be selected into an inventory. The criteria inventory is useful by providing what to represent in simplified model. 

% The simplified model will be developed based on real train information in order to natively support Dutch designers. The result simplified model output shall be in cooperation with the criteria inventory made in previous steps.

% Using the developed model and the knowledge of literature study, an imaginary steel railway bridge can be designed, in order to verify the reliability of the newly developed tool. 


The main goal of this thesis is to discuss the sufficiency and reliability of current criteria regarding lateral dynamics of steel railway bridges that exist in Eurocode 1991-2. Original reports where criteria were proposed will be reviewed to reveal the background of them to Eurocode users. 

Depending on the output of previous step, if the current requirements/criteria show lack of reliability/sufficiency of Eurocode 1991-2, further recommendation will be made for revising or amending current Eurocode.

This report also aims to providing knowledge of railway bridge dynamics to Eurocode users. The dynamics of railway bridge concerns cross-field knowledge, including intelligence of railway tracks/ vehicles/ bridge structures, etc. Some of them are essential to dynamics problem but unfamiliar to Eurocode users. Thus the report is willing to convey users the reason and importance of dynamic concerns of railway bridges.



\section{Main steps}

In order to provide a better tool for designers when they encounter lateral dynamic problems on steel railway bridges, following objectives are made:

\begin{enumerate}

\item Literature research of dynamic actions as well as their criteria on railway bridges, rails and train vehicles in order to give a better understanding of the background of the criteria which is unclear to the designers. Study the dynamic behaviour of these system respectively. Then discuss their effects when combined.

\item Find the original reports which proposed 1.2Hz criterion but assess the content of these reports.

\item Develop a method to check if a bridge is prone to encounter dynamic problems. The method should be simple. It should be compatible with FEM software and give suggestions for further bridge modification. However form of the method will depend on output of the previous research output.
 

\end{enumerate}

\section{Research methodology}
\section{Outline of the report}

