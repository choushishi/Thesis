%!TEX root = first try.tex

\chapter{INTRODUCTION}

\section{Summary of topic}

The lateral dynamic behaviour of steel railway bridges are minimally discussed in Eurocodes and designers lack knowledge of background of criteria proposed in the code. For example, there is one criterion in Eurocode requiring railway bridges should have a lateral natural frequency higher than 1.2Hz. However, this criterion is becoming more and more unsuitable because longer span provides lower natural frequencies. For bridges having span more than 100m, it is almost guaranteed that the first lateral natural frequency of the bridge fall below 1.2Hz, unable to meet the requirement of Eurocode. 

Criteria on lateral dynamics of railway bridges are complicated if taking vehicle systems and interaction into account. Designers need a better knowledge on railway dynamics and a tool in calculating the lateral dynamic behaviour of the whole system. This tool needs to be simple to meet the engineering needs.  

\section{Motivation of the thesis}


\section{Objectives and research question}\label{sec:introduction}

The main goal is to think of a method to verify whether a bridge is expected to encounter transverse dynamic problems. 

In the literature study the criteria of different systems involved in dynamic response of steel railway bridges will be investigated. Governing criteria will be selected into an inventory. The criteria inventory is useful by providing what to represent in simplified model. 

The simplified model will be developed based on real train information in order to natively support Dutch designers. The result simplified model output shall be in cooperation with the criteria inventory made in previous steps.

Using the developed model and the knowledge of literature study, an imaginary steel railway bridge can be designed, in order to verify the reliability of the newly developed tool. 

\section{Main steps}

In order to provide a better tool for designers when they encounter lateral dynamic problems on steel railway bridges, following objectives are made:

\begin{enumerate}

\item Literature research of the background of the criteria of railway bridges, rails and train vehicles in order to give a better understanding of the background of the criteria which is unclear to the designers.  Create an inventory of criteria on lateral dynamic by selecting criteria among systems mentioned above. 

\item Develop a method to check if a bridge is prone to encounter dynamic problems.

Several hypothesis have been made and will be illustrated during kick-off presentation.

\item Verify the model developed in the previous step by checking a long-span railway bridge. 

\end{enumerate}

\section{Research methodology}
\section{Outline of the report}

