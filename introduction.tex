% -*- root: first try.tex -*-

%!TEX root = first try.tex

\chapter{INTRODUCTION}
Dear Prof.Frans Bijlaard

This is a sample thesis created by Sijian Deng.

General remarks by Sijian: Firstly I want to apologize for flaws in the whole document, I know for some part it's not that pleasant to read but I'll fix them in future versions. Since everything written here is not finally decided, please read Section\ref{sec:introduction} first for my primarily thought about the objectives. In addition: I don't want to narrow down on one particular topic before the first kick-off meeting. 

Since I'm still open for different interesting topics with regard to railway bridge dynamics, please let me know if you know any possible topic besides from those listed in Section\ref{sec:introduction} for me. 

The table of contents will give you a grasp of my thesis structure. Although there is nothing besides from literature research for now, I already have a vague thought about following thesis contents. Please take a look at the sections after literature research. 

\section{Problem statement}
\section{Motivation of the thesis}
\section{Objectives and research question}\label{sec:introduction}

% The main goal of this thesis is to discuss the dynamic behaviour of long-span steel railway bridges. The reason to study long-span bridges' dynamic characteristics is because Eurocodes lack related critera/principles on bridges with span over 100m, resulting in higher possibility for designers to invest a lot of time and money into FEM modelling in order to make sure the design is conservative enough. To provide designers with more knowledge of long-span bridges dynamics, following objectives are formed:

% -A literature study aims to research the state of art of dynamic knowledge on steel railway bridges. 

% -Filter out possible criteria for long-span bridges that's not included in Eurocode. Assess the reliability of these criteria.

% - Modify current available dynamic analysing method for the usage of long-span bridges.

% - A case study on real bridge. Evaluate the criteria and analysing method proposed in previous chapter by comparing them with the result given by FEM analyses.

% \textit{To summarise the objectives of this master's thesis, the following research proposal was formed:}

% \textit{\textbf{This master's thesis is going to investigate the dynamic behaviour of long-span steel railway bridge and its relating designing criteria }}

Within the study field of steel railway bridge dynamics, I found several potential research topics as my literature research goes on. These topics are seldom(or never) discussed according to my knowledge. I want to ask your opinion on these topics and tell me whether it's feasible for my graduation thesis. 

\begin{enumerate}
	\item Criteria for bridge lateral behaviour. 1.2Hz sufficient or not? Or is it too conservative? Any other criteria can be found, to be used together with 1.2Hz? Please see Section\ref{sec:Transverse-deformations-and-vibrations} for detailed information about this criteria.
	\item Long span bridges - usually okay with dynamic - but always have frequency below 1.2Hz->can't meet Eurocode requirements. Any possible situation to avoid dynamic check?
	\item Upgrade train dynamic signature method for dynamic analyses - expand application from vertical direction only to 3-dimensional. Please see Seciton \ref{sec:tds} for detailed information.
	\item Background of the verification checks
	\item Study the future usage of high-strength steel on railway bridges - less mass, same stiffness - higher possibility of resonance under train excitation
\end{enumerate}


\section{Research methodology}
\section{Outline of the report}

