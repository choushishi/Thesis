%!TEX root = first try.tex


\chapter{INTRODUCTION}

\section{Summary of topic}


Lateral railway bridge dynamics is an engineering topic that related to both bridge structure and operating railway vehicles. There is little knowledge about lateral dynamics of railway bridges among structural engineers, but it is only until recently this topic catches focus. The reason is that an Eurocode criterion regarding the topic is frequently rejecting new long span railway bridge designs. 

Although there's no records of bridge/vehicle failure due to lateral railway vehicle dynamic loading, it is found during the research process of this thesis that ERRI(former UIC) had already done investigation on this topic before 1994 and presented the results in D181 report series. This information of D181 is hardly known to both researchers and engineers after Eurocode EN1991-2's publication in 1994. However, several criterion proposed by D181 committee were adopted in EN1991-2.

Throughout D181 report series and EN1991-2, it can been seen that almost no attention is put on lateral dynamics of long span railway bridges. This thesis aims to review both documents and conduct further researches on the topic and propose a simplier 'engineering' verification method of lateral dynamics of railway bridges.

\section{Motivation of the thesis}

The lateral dynamic behaviour of steel railway bridges are minimally discussed in Eurocodes and designers lack knowledge of background of the criteria proposed in the code. For example, there is one criterion in Eurocode requiring railway bridges should have a lateral natural frequency higher than 1.2Hz. However, this criterion is becoming more and more unsuitable for modern bridge designs because longer span provides lower natural frequencies. For bridges having span longer than 100m, it is almost guaranteed that the first lateral natural frequency of the bridge is lower than 1.2Hz, and unable to meet the requirement of EN1991-2. 

Since for most bridge designs, changing the bridge parameters to meet the 1.2Hz criterion isn't a viable option because it's expensive to add stiffness to the bridge, the projects need a solution to solve the problem. To gain more knowledge on the topic, a research on the background of 1.2Hz criterion is initiated by Iv-Infra, presented in this report. The discovery in the background research stimulates further researches including domestic lateral wavelength of trains in the Netherlands, lateral force on tracks, simplier engineering verification method, etc. 

\section{Objectives and research question}\label{sec:introduction}


The report has following objectives.

\begin{enumerate}[-]
\item Provide insight into lateral dynamics of railway bridges. Since the general knowledge about the 


\item Find the source of 1.2Hz criterion. Interpret its original intention by reviewing the original documents. Give a conclusion on the correctness of 1.2Hz criterion based on the interpretation.
\item Find alternative strategy for validating the lateral dynamics of railway bridges
\item Develop a practical method based on newly found strategy 
\end{enumerate}

\vspace*{0.5cm}

The research question can be summarized as:

\vspace*{0.5cm}

\textit{Conclude the correctness of current existing criterion on lateral railway bridge dynamics and develop new practical method for checking lateral dynamics of railway bridges based on an alternative checking strategy. }


\section{Main steps}

In order to carry out the objectives and research questions, the thesis will be conducted in following steps:


\begin{enumerate}

\item Literature research on the knowledge regarding lateral railway bridge dynamics. This section aims to provide insight to readers so emphasis is placed on fundamental information. Since general dynamics of the bridges are widely discussed and the only thing that differs is that railway bridges are subjected to railway vehicle loads, literature research also focuses on conveying basic knowledge of dynamics of railway vehicles.

\item Find the original reports that proposed the 1.2Hz criterion and assess the content of these reports. Further investigate the research process and conclusions of these reports to yield more useful information for following researches.

\item Using the knowledge in the original reports, study the dynamics properties of trains running in the Netherlands.

\item Develop a practical method for simple calculating the time-history of the dynamic response of the bridge under train vehicle load on the basis of numerical simulation results provided by D181 committee in their report series.

\item Apply above developed practical method on a real project to finalize the method for engineering usage and as well as illustrates the usage of the method.

\end{enumerate}

\newpage
\section{Outline of the report}


\begin{figure}[h!]
\centering
\begin{tikzpicture}[
    node distance = 5mm and 4mm,
    point/.style={circle, inner sep =0pt, minimum size = 1pt, fill =black}]
    \matrix[row sep = 5mm, column sep =4 mm] {
        \node[punkt] (Literature) {Literature research}; & & 
        \node(dummy1) [point] {}; & \\
        \node[punkt] (1991)  {Investigation of EN1991-2}; &
        \node[punkt] (D181) {Investigation of D181 report series}; &
        \node[punkt] (Wavelength) {Lateral Wavelength of Dutch Vehicles}; &
        \node[punkt] (Practical) {Practical method for checking lateral dynamics of railway bridges}; \\
        & \node [point] (dummy3) {}; &
         &
        \node [point] (dummy5) {}; \\
    };
    
    \graph {
        (Literature) -> (1991) ;
        (Literature) -- (dummy1) -> (Wavelength);
        (1991) -> (D181) -> (Wavelength) -> (Practical);
        (D181) -- (dummy3) -- (dummy5) -> (Practical);
    };
\end{tikzpicture}
\caption{Logic relationship of chapters}
\label{fig:logicchpater}
\end{figure}

The logic relationship of selected chapters is illustrated in Figure.\ref{fig:logicchpater}

The report is created in following structure:

\begin{enumerate}
    \item Literature study on basic concepts of railway bridge dynamics.
    \item Deep investigation of EN1991-2 supporting research report series
    \item Wavelength study
    \item Introduction to the analytical method
    \item Development of the new practical checking method by adopting the analytical model and outputs of VAMPIRE simulations
    \item Suggestions for Eurocode
    \item Conclusions
\end{enumerate}
